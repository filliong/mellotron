%!TEX root = mellotron.tex

\chapter{Physical Model}

\section{Classical Physics}

Charged particles experience a force when plunged into an electromagnetic field.
Classically, this is simply the Lorentz force, which reads, in SI units,
  \begin{equation}
    \dv{p^\mu}{s} = \frac{q}{m}F^{\mu\nu}p_\nu
  \end{equation}
where $p^\mu$ is the 4-momentum, $s$ the proper time, $q$ the charge of the
particle, $m$ its mass and $F^{\mu\nu}$ is the electromagnetic field tensor, i.e.
  \begin{align}
    F^{\mu\nu} &= \partial^\mu A^\nu - \partial^\nu A^\mu, \\
               &= \begin{bmatrix}
                    0      & -E_1/c     & -E_2/c     & -E_3/c \\
                    E_1/c  & 0          & -B_3       & B_2     \\
                    E_2/c  & B_3        & 0          & -B_1    \\
                    E_3/c  & -B_2       & B_1        & 0
                  \end{bmatrix}.
  \end{align}
We use the mostly minus sign convention $(+---)$.

\section{Radiation Reaction}

To model radiation reaction, we employ the Landau-Lifshitz equation, which
reads, in SI units,
  \begin{equation}
    \dv{p^\mu}{s} = \frac{q}{m}F^{\mu\nu}p_\nu
                   +\frac{q^2}{6\pi\epsilon_0m^3c^3}
                      \left[
                        q\left(\partial_\alpha F^{\mu\nu}\right)p^\alpha p_\nu
                      - q^2 F^{\mu\nu}F_{\alpha\nu}p^\alpha
                      + \frac{q^2}{m^2c^2} F^{\alpha\nu}p_\nu F_{\alpha\lambda}p^\lambda p^\mu
                      \right].
  \end{equation}

In the implementation, we assume QED units. The equation thus reads
  \begin{equation}
    \dv{p^\mu}{s}=\frac{q}{e}\frac{m_e}{m}F^{\mu\nu}p_\nu
                   +\frac{2\alpha}{3}\left(\frac{fm_e}{m}\right)^3
                      \left[
                        \left(\partial_\alpha F^{\mu\nu}\right)p^\alpha p_\nu
                      - fF^{\mu\nu}F_{\alpha\nu}p^\alpha
                      + f\left(\frac{m_e}{m}\right)^2 F^{\alpha\nu}p_\nu F_{\alpha\lambda}p^\lambda p^\mu
                      \right]
\end{equation}
where $f$ is the fractional charge of the particle $q=fe$.

It can be shown \cite{} that the third term is dominant. We thus neglect the other
two in the implementation. We thus solve the following differential equation
  \begin{equation}
    \dv{p^\mu}{s}=\frac{q}{e}\frac{m_e}{m}F^{\mu\nu}p_\nu
                   +\frac{2\alpha f^4}{3}\left(\frac{m_e}{m}\right)^5
                   F^{\alpha\nu}p_\nu F_{\alpha\lambda}p^\lambda p^\mu.
  \end{equation}
