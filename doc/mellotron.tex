% --------------------------------------------------------------------------- %
% Author:          Joey Dumont                <joey.dumont@gmail.com>         %
% Date created:    2016-09-29                                                 %
% Description:     Document detailing MELLOTRON.                              %
% License:         CC0 1.0 Universal                                          %
%                  <https://creativecommons.org/publicdomain/zero/1.0/>       %
% --------------------------------------------------------------------------- %

\documentclass[10pt,letterpaper,extrafontsizes, onecolumn,openright]{memoir}

% ----------------------------------------------------------------- %
% --                          Packages                           -- %
% ----------------------------------------------------------------- %
\usepackage{graphicx}
\usepackage{subfig}
\usepackage{physics}
\usepackage{amsmath,amssymb,mathtools,esint}
\usepackage[charter]{mathdesign}
\usepackage{multicol}
\usepackage[per-mode=symbol,inter-unit-product = {}\cdot{}]{siunitx}
\usepackage[square,numbers,sort&compress,merge,elide]{natbib}

\DeclareMathOperator{\sign}{sgn}
\newcommand{\sgn}[1]{\sign\left(#1\right)}

% ----------------------------------------------------------------- %
% --                         Page Layout                         -- %
% ----------------------------------------------------------------- %

\setlrmarginsandblock{0.2\paperwidth}{*}{1} % Left and right margin
\setulmarginsandblock{0.2\paperwidth}{*}{1} % Upper and lower margin
\checkandfixthelayout

% ----------------------------------------------------------------- %
% --                     Sectional Divisions                     -- %
% ----------------------------------------------------------------- %

\maxsecnumdepth{subsection} % Subsections (and higher) are numbered
\setsecnumdepth{subsection}

\makeatletter %
\makechapterstyle{standard}{
  \setlength{\beforechapskip}{0\baselineskip}
  \setlength{\midchapskip}{1\baselineskip}
  \setlength{\afterchapskip}{8\baselineskip}
  \renewcommand{\chapterheadstart}{\vspace*{\beforechapskip}}
  \renewcommand{\chapnamefont}{\centering\normalfont\Large}
  \renewcommand{\printchaptername}{\chapnamefont \@chapapp}
  \renewcommand{\chapternamenum}{\space}
  \renewcommand{\chapnumfont}{\normalfont\Large}
  \renewcommand{\printchapternum}{\chapnumfont \thechapter}
  \renewcommand{\afterchapternum}{\par\nobreak\vskip \midchapskip}
  \renewcommand{\printchapternonum}{\vspace*{\midchapskip}\vspace*{5mm}}
  \renewcommand{\chaptitlefont}{\centering\bfseries\LARGE}
  \renewcommand{\printchaptertitle}[1]{\chaptitlefont ##1}
  \renewcommand{\afterchaptertitle}{\par\nobreak\vskip \afterchapskip}
}
\makeatother

\chapterstyle{standard}

\setsecheadstyle{\normalfont\large\bfseries}
\setsubsecheadstyle{\normalfont\normalsize\bfseries}
\setparaheadstyle{\normalfont\normalsize\bfseries}
\setparaindent{0pt}\setafterparaskip{0pt}

% ----------------------------------------------------------------- %
% --                     Floats and Captions                     -- %
% ----------------------------------------------------------------- %

\makeatletter                        % You do not need to write [htpb] all the time
\renewcommand\fps@figure{htbp}       %
\renewcommand\fps@table{htbp}        %
\makeatother                         %


\captiondelim{\space }               % A space between caption name and text
\captionnamefont{\small\bfseries}    % Font of the caption name
\captiontitlefont{\small\normalfont} % Font of the caption text

%\changecaptionwidth                 % Change the width of the caption
%\captionwidth{1.0\textwidth}        %

% ----------------------------------------------------------------- %
% --                          Abstract                           -- %
% ----------------------------------------------------------------- %

\renewcommand{\abstractnamefont}{\normalfont\small\bfseries} % Font of abstract title
\setlength{\absleftindent}{0.1\textwidth} % Width of abstract
\setlength{\absrightindent}{\absleftindent}

% ----------------------------------------------------------------- %
% --                      Header and Footer                      -- %
% ----------------------------------------------------------------- %

\makepagestyle{standard}      % Make standard pagestyle

\makeatletter                 % Define standard pagestyle
\makeevenfoot{standard}{}{}{} %
\makeoddfoot{standard}{}{}{}  %
\makeevenhead{standard}{\bfseries\thepage\normalfont\qquad\small\leftmark}{}{}
\makeoddhead{standard}{}{}{\small\rightmark\qquad\bfseries\thepage}
% \makeheadrule{standard}{\textwidth}{\normalrulethickness}
\makeatother                  %

\makeatletter
\makepsmarks{standard}{
\createmark{chapter}{both}{shownumber}{\@chapapp\ }{ \quad }
\createmark{section}{right}{shownumber}{}{ \quad }
\createplainmark{toc}{both}{\contentsname}
\createplainmark{lof}{both}{\listfigurename}
\createplainmark{lot}{both}{\listtablename}
\createplainmark{bib}{both}{\bibname}
\createplainmark{index}{both}{\indexname}
\createplainmark{glossary}{both}{\glossaryname}
}
\makeatother

\makepagestyle{chap}          % Make new chapter pagestyle

\makeatletter
\makeevenfoot{chap}{}{\small\bfseries\thepage}{} % Define new chapter pagestyle
\makeoddfoot{chap}{}{\small\bfseries\thepage}{}  %
\makeevenhead{chap}{}{}{}
\makeoddhead{chap}{}{}{}
% \makeheadrule{chap}{\textwidth}{\normalrulethickness}
\makeatother

\nouppercaseheads
\pagestyle{standard}               % Choosing pagestyle and chapter pagestyle
\aliaspagestyle{chapter}{chap} %

% ----------------------------------------------------------------- %
% --                        New Commands                         -- %
% ----------------------------------------------------------------- %

\newcommand{\p}{\ensuremath{\partial}}
\newcommand{\bo}[1]{\ensuremath{\boldsymbol{#1}}}
\newcommand{\bou}[1]{\ensuremath{\hat{\boldsymbol{#1}}}}
\renewcommand{\Re}[1]{\ensuremath{\text{Re}\left\{#1\right\}}}

% -- Tag text in equations.
\makeatletter
\providecommand\add@text{}
\newcommand\tagaddtext[1]{%
  \gdef\add@text{#1\gdef\add@text{}}}%
\renewcommand\tagform@[1]{%
  \maketag@@@{\llap{\add@text\quad}(\ignorespaces#1\unskip\@@italiccorr)}%
}
\makeatother

% ----------------------------------------------------------------- %
% --                      Table of Contents                      -- %
% ----------------------------------------------------------------- %

\maxtocdepth{subsection} % Only parts, chapters and sections in the table of contents
\settocdepth{subsection}

\AtEndDocument{\addtocontents{toc}{\par}} % Add a \par to the end of the TOC

% ----------------------------------------------------------------- %
% --                     Internal Hyperlinks                     -- %
% ----------------------------------------------------------------- %

\usepackage{hyperref}   % Internal hyperlinks
\hypersetup{
  pdfborder={0 0 0},       % No borders around internal hyperlinks
  pdfauthor={Joey Dumont}, % author
  pdftitle={MELLOTRON}
}

% ----------------------------------------------------------------- %
% --                          Document                           -- %
% ----------------------------------------------------------------- %

\begin{document}

\frontmatter
\pagestyle{empty}
%\maketitle

% title page
%\vspace*{\fill}
\begin{vplace}
\begin{center}
\LARGE\textsf{C++ Library for the Classical Dynamics  \\ of Charged Particles}\par
\end{center}
\begin{center}
\HUGE\textsf{\textsc{mellotron}}\par
\end{center}
%\begin{center}
%\LARGE\textsf{is a}\par
%\end{center}


\begin{center}
\Huge\textsf{User Guide}\par
\end{center}
\begin{center}
\LARGE\textsf{Joey Dumont}\par
\bigskip
\normalsize\textsf{Version 0.0.0}\par
\end{center}
\end{vplace}
%\vspace*{\fill}
%\def\THP{T\kern-0.2em H\kern-0.4em P}%   OK for CMR
%\def\THP{T\kern-0.15em H\kern-0.3em P}%   OK for Palatino
%\newcommand*{\THPress}{The Herries Press}%
%\begin{center}
%\settowidth{\droptitle}{\textsf{\THPress}}%
%\textrm{\normalsize \THP} \\
%\textsf{\THPress} \\[0.2\baselineskip]
%\includegraphics[width=\droptitle]{anvil2.mps}
%\setlength{\droptitle}{0pt}%
%\end{center}
\cleardoublepage

\begin{abstract}

\end{abstract}

\cleardoublepage

\tableofcontents*
\clearpage

\mainmatter
\pagestyle{standard}

% -- Introduction chapter.
\chapter{Introduction}

% -- Installation instructions.
\chapter{Installation}

% -- Mathematical background of the diffraction formulae.
\chapter{Physical Model}

\section{Classical Physics}

Charged particles experience a force when plunged into an electromagnetic field.
Classically, this is simply the Lorentz force, which reads, in SI units,
  \begin{subequations}
  \begin{align}
    \dv{x^\mu}{s} &= \frac{1}{m}p^\mu \\
    \dv{p^\mu}{s} &= \frac{q}{m}F^{\mu\nu}p_\nu
  \end{align}
  \end{subequations}
where $p^\mu$ is the 4-momentum, $s$ the proper time, $q$ the charge of the
particle, $m$ its mass and $F^{\mu\nu}$ is the electromagnetic field tensor, i.e.
  \begin{subequations}
  \begin{align}
    F^{\mu\nu} &= \partial^\mu A^\nu - \partial^\nu A^\mu, \\
               &= \begin{bmatrix}
                    0      & -E_1/c     & -E_2/c     & -E_3/c \\
                    E_1/c  & 0          & -B_3       & B_2     \\
                    E_2/c  & B_3        & 0          & -B_1    \\
                    E_3/c  & -B_2       & B_1        & 0
                  \end{bmatrix}.
  \end{align}
  \end{subequations}
We use the mostly minus sign convention $(+---)$.

\section{Radiation Reaction}

To model radiation reaction, we employ the Landau-Lifshitz equation, which
reads, in SI units,
  \begin{equation}
    \dv{p^\mu}{s} = \frac{q}{m}F^{\mu\nu}p_\nu
                   +\frac{q^2}{6\pi\epsilon_0m^3c^3}
                      \left[
                        q\left(\partial_\alpha F^{\mu\nu}\right)p^\alpha p_\nu
                      - q^2 F^{\mu\nu}F_{\alpha\nu}p^\alpha
                      + \frac{q^2}{m^2c^2} F^{\alpha\nu}p_\nu F_{\alpha\lambda}p^\lambda p^\mu
                      \right].
  \end{equation}

It can be shown \cite{} that the third term dominates over the two other terms.
The equation thus reads
  \begin{equation}
    \dv{p^\mu}{s} = \frac{q}{m}F^{\mu\nu}p_\nu
                   +\frac{q^4}{6\pi\epsilon_0m^5c^5}F^{\alpha\nu}p_\nu F_{\alpha\lambda}p^\lambda p^\mu.
  \end{equation}

It will be useful to work in a system of units where the fields are normalized with
respect to its characteristic wavelength and with respect to the charge and mass
of the elctron. Given that $\omega_0$ is a characteristic frequency of the field,
we take
  \begin{subequations}
  \begin{align}
    \vb{E}' &= \frac{|e|}{m_e\omega_0 c} \vb{E} & \vb{B}' &= \frac{|e|}{m_e\omega_0}\vb{B} \\
    \vb{p}' &= \frac{\vb{p}}{m_ec}              & \vb{x}' &= \frac{\omega_0}{c}\vb{x}    \\
    t'      &= \omega_0 t.
  \end{align}
  \end{subequations}
Substituting this into the Landau-Lifshitz equation yields
  \begin{align}
  	\dv{x^\mu}{t'}   &= \frac{p^\mu}{\gamma m_\text{el}} \\
    \dv{p^0}{t'}     &= \frac{q_\text{el}}{\gamma m_\text{el}}\vb{p}'\cdot\vb{E}'
                       +\frac{2\alpha}{3\gamma}\frac{q_\text{el}^4}{m_\text{el}^5}\frac{\hbar\omega_0}{m_ec^2}
                      \left[(\vb{p}'\cdot\vb{E}')^2-(\gamma\vb{E}'+\vb{p}'\times\vb{B}')^2\right]p^0 \\
    \dv{\vb{p}'}{t'} &= q_\text{el}\vb{E}' +\frac{q_\text{el}}{\gamma m_\text{el}}\vb{p}'\times\vb{B}'
                      +\frac{2\alpha}{3\gamma}\frac{q_\text{el}^4}{m_\text{el}^5}\frac{\hbar\omega_0}{m_ec^2}
                      \left[(\vb{p}'\cdot\vb{E}')^2-(\gamma\vb{E}'+\vb{p}'\times\vb{B}')^2\right]\vb{p}'
  \end{align}
where $q_\text{el}$ is the charge of the particle in units of the electron charge.


% In the implementation, we assume QED units. The equation thus reads
%   \begin{equation}
%     \dv{p^\mu}{s}=\frac{q}{e}\frac{m_e}{m}F^{\mu\nu}p_\nu
%                    +\frac{2\alpha}{3}\left(\frac{fm_e}{m}\right)^3
%                       \left[
%                         \left(\partial_\alpha F^{\mu\nu}\right)p^\alpha p_\nu
%                       - fF^{\mu\nu}F_{\alpha\nu}p^\alpha
%                       + f\left(\frac{m_e}{m}\right)^2 F^{\alpha\nu}p_\nu F_{\alpha\lambda}p^\lambda p^\mu
%                       \right]
% \end{equation}
% where $f$ is the fractional charge of the particle $q=fe$.

% It can be shown \cite{} that the third term is dominant. We thus neglect the other
% two in the implementation. We thus solve the following differential equation
%   \begin{equation}
%     \dv{p^\mu}{s}=\frac{q}{e}\frac{m_e}{m}F^{\mu\nu}p_\nu
%                    +\frac{2\alpha f^4}{3}\left(\frac{m_e}{m}\right)^5
%                    F^{\alpha\nu}p_\nu F_{\alpha\lambda}p^\lambda p^\mu.
%   \end{equation}

% It is more useful to work with the laboratory frame time where $\gamma ds = dt$.
% We have
%   \begin{equation}
%     \dv{p^\mu}{t}=\frac{1}{\gamma}\frac{q}{e}\frac{m_e}{m}F^{\mu\nu}p_\nu
%                    +\frac{2\alpha f^4}{3\gamma}\left(\frac{m_e}{m}\right)^5
%                    F^{\alpha\nu}p_\nu F_{\alpha\lambda}p^\lambda p^\mu.
%   \end{equation}

\section{Explicit Expressions}
We provide explicit expressions of the differential equations. We proceed term
by term. The first term is the Lorentz force and can easily be written as
  \begin{align*}
    g_1^\mu &= F^{\mu\nu}p_\nu                  &  \\
    g_1^0   &= F^{0\nu}p_\nu                    & g_1^i   &= F^{i\nu}p_\nu    \\
            &= F^{0i}p_i                       &         &= F^{i0}p_0+F^{ij}p_j        \\
            &= \left(\vb{p}\cdot\vb{E}\right)   &         &= \frac{\epsilon E_i}{c}-\epsilon^{ijk}p_jB_k \\
            &                                   &         &= \frac{\epsilon E_i}{c}+\left(\vb{p}\times\vb{B}\right)_i
  \end{align*}

The second term yields
  \begin{align*}
    g_2^\mu &= \left(\partial_\alpha F^{\mu\nu}\right)p^\alpha p_\nu & \\
    g_2^0   &= \left(\partial_\alpha F^{0\nu}\right)p^\alpha p_\nu   &
  \end{align*}

The fourth is simply the squared magnitude of the Lorentz force vector, i.e.
  \begin{equation}
    g_4^\mu = \left[\left(\frac{\epsilon\vb{E}}{c}+\left(\vb{p}\times\vb{B}\right)\right)^2-(\vb{p}\cdot\vb{E})^2\right] p^\mu.
  \end{equation}

\section{Analytical Solutions}

To test the implementation, we use an analytical solution in a uniform electrostatic field,
neglecting radiation reaction.
The vector part of the momentum equation reads
	\begin{equation}
		\dv{\vb{p}}{t} = q_\text{el}\vb{E}
	\end{equation}
and can readily be integrated to yield
	\begin{equation}
		p(t) = q_\text{el}\vb{E}t+\vb{p}_0.
	\end{equation}
The time component, $p^0=m_\text{el}\gamma$ reads
	\begin{align}
		\dv{\gamma}{t} &= \frac{q_\text{el}}{\gamma m_\text{el}^2}\vb{p}\cdot\vb{E} \\
		               &= \frac{q_\text{el}}{\gamma m_\text{el}^2}\left(q_\text{el}\vb{E}^2 t + \vb{p}_0\cdot\vb{E}\right)
	\end{align}
Which yields
	\begin{equation}
		\gamma(t) = \sqrt{\frac{q_\text{el}^2\vb{E}^2}{ m_\text{el}^2}t^2+2\frac{q_\text{el}}{m_\text{el}^2}\vb{p}_0\cdot\vb{E}t+k_1}
	\end{equation}
We can fix the constant by asking that $\gamma(0)=\sqrt{1+\vb{p}_0^2/m_\text{el}^2}$.
	\begin{align}
		\gamma(t) &= \sqrt{\frac{q_\text{el}^2\vb{E}^2}{ m_\text{el}}t^2+2\frac{q_\text{el}}{ m_\text{el}}\vb{p}_0\cdot\vb{E}t+1+\vb{p}_0^2/m_\text{el}^2}\\
		          &= \sqrt{1+\frac{1}{m_\text{el}^2}\left({p_0}^2+2q_\text{el}\vb{p}_0\cdot\vb{E}t+q_\text{el}^2\vb{E}^2t^2\right)} \\
		          &= \sqrt{1+\vb{p}(t)^2/m_\text{el}^2},
	\end{align}
as it should be.

We can now solve for the position as a function of time. We have
	\begin{align}
		\dv{\vb{x}}{t} &= \frac{1}{\gamma m_\text{el}}\left(q_\text{el}\vb{E}t+\vb{p}_0\right) \\
		               &= \frac{1}{m_\text{el}}\frac{q_\text{el}\vb{E}t+\vb{p}_0}{\sqrt{1+\frac{1}{m_\text{el}^2}\left({p_0}^2+2q_\text{el}\vb{p}_0\cdot\vb{E}t+q_\text{el}^2\vb{E}^2t^2\right)}}
	\end{align}
which has the solution
	\begin{align}
		\dv{\vb{x}}{t} &= \frac{\gamma\vb{E}}{\vb{E}^2}\frac{m_\text{el}}{q_\text{el}}+c_1 \\
		               &= \frac{\gamma\vb{E}}{\vb{E}^2}\frac{m_\text{el}}{q_\text{el}}
		                 +\vb{x}_0-\frac{\gamma(0)\vb{E}}{\vb{E}^2}\frac{m_\text{el}}{q_\text{el}}.
  \end{align}

We repeat the process for a magnetic field in the $z$ direction, i.e.
$\vb{B}=(0,0,B_z)$. In this case, the $\gamma$ factor is constant in time,
and it equal to its initial value. We can write the solution
	\begin{align}
		p_x(t) &= p_\perp \sin(\Omega t + \theta_0) \\
		p_y(t) &= p_\perp \cos(\Omega t + \theta_0) \\
		p_z(t) &= p_z(0)
	\end{align}
where $p_\perp=\sqrt{p_x(0)^2+p_y(0)^2}$, $\Omega=q_\text{el}B_z/(\gamma m_\text{el})$
and $\theta_0=\arctan(p_x(0)/p_y(0))$. The position as a function of time
	\begin{align}
		x(t)  &= -\frac{p_\perp}{\gamma m_\text{el}\Omega}\left(\cos(\Omega t+\theta_0)-\cos(\theta_0)\right)+x_0 \\
		y(t)  &=  \frac{p_\perp}{\gamma m_\text{el}\Omega}\left(\sin(\Omega t+\theta_0)-\sin(\theta_0)\right)+y_0 \\
		z(t)  &= \frac{p_z(0)}{\gamma m_\text{el}}t+z_0.
	\end{align}

\chapter{Documentation of Drivers}

\section{Drivers based on the StrattoCalculator}

\subsection{Chirp}

Chirp is considered in the frequency domain, where it is simply given by
a pure phase for each frequency component, and in turn for each component
of the incident field:

\begin{align}
  \vb{E}(\vb{r},k) &\mapsto \vb{E}(\vb{r},k)e^{i\phi(\omega)} \\
  \vb{B}(\vb{r},k) &\mapsto \vb{B}(\vb{r},k)e^{i\phi(\omega)}.
\end{align}

Since well-behaved beams have well-behaved spectral phases (most of the time),
we expand $\phi(\omega)$ about some central frequency $\omega_c$ and express
it as
  \begin{equation}
    \phi(\omega) = \sum_i \phi_i \frac{(\omega-\omega_c)^i}{i!}.
  \end{equation}
In the MELLOTRON drivers, we take as input $\omega_c$ and a vector of $\phi_i$
in SI units. Note that the $\phi_n$ will have units of $[\text{second}]^n$.


\appendix

\bibliographystyle{osajnl}
\bibliography{mellotron_ref.bib}


\end{document}
%
% ****** End of file template.aps ******
